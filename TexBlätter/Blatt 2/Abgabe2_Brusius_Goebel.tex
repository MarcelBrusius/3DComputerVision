\documentclass[11pt,english,paper=a4]{scrartcl}

%AMS-Pakete
\usepackage{amsmath,amsfonts,makeidx,amsthm,amssymb}
\usepackage{graphicx}
\usepackage[abs]{overpic}
\usepackage[utf8]{inputenc}
\usepackage[english]{babel}
\usepackage[T1]{fontenc}
\usepackage{array}
\usepackage{float}
\usepackage{framed}
\usepackage{caption}
\usepackage{subcaption}
\usepackage{hyperref}
\usepackage{tikz}
\usepackage[section]{algorithm}
\usepackage{algorithmicx}
\usepackage[noend]{algpseudocode}
\usepackage{listings}
\lstset{language=Matlab}
\usepackage{textcomp}
\usepackage{pgfplots}
\usepackage{mathtools}
\usepackage{enumitem}
\usepackage{xcolor}
\usepackage{tabularx}
\definecolor{tuklblue}{RGB}{0,95,140}
\definecolor{tuklred}{RGB}{185,40,25}
\definecolor{tuklgreen}{RGB}{1,147,119}
\hypersetup{%
	colorlinks=true,%
	linkcolor=tuklblue,%
	citecolor=tuklgreen,%
	urlcolor=tuklred%
}
\renewcommand{\algorithmicrequire}{\textbf{Input:}}
\renewcommand{\algorithmicensure}{\textbf{Output:}}

\setlength{\extrarowheight}{1ex}
\theoremstyle{definition} %So ist alles nicht kursiv. Sätze und Definitionen wären kursiv eventuell schöner. Wie geht das genau nur diese zu ändern?
\newtheorem{definition}{Definition}[section]
\newtheorem{theorem}[definition]{Theorem}
\newtheorem{lemma}[definition]{Lemma}
\newtheorem{corollary}[definition]{Corollary}
\newtheorem{notation}[definition]{Notation}
\newtheorem{remark}[definition]{Remark}
\newtheorem{algo}[algorithm]{Algorithm}
\newtheorem{proposition}[definition]{Proposition}
\newtheorem{example}[definition]{Example}
\newtheorem{Exercise}{Exercise}
\newtheorem*{Solution}{Solution}

\makeatletter 
\let\c@algorithm\c@definition
\makeatother


\newcommand{\zb}[1]{\ensuremath{\boldsymbol{#1}}}
\newcommand{\R}{\mathbb{R}}
\DeclareMathOperator*{\argmin}{argmin}
\DeclareMathOperator*{\argmax}{argmax}
\DeclareMathOperator*{\grad}{grad}
\DeclareMathOperator*{\supp}{supp}
\DeclareMathOperator{\spn}{span}
\newcommand{\tT}{\mathrm{T}}

\setlength{\parindent}{0pt}
\setcounter{MaxMatrixCols}{11}

\begin{document}
	
	\begin{titlepage}
		\parskip=0pt
		\parindent=0pt
		\begin{tikzpicture}[remember picture,overlay]
		\node [anchor=north west, inner xsep=0pt, inner ysep=2.1cm] at
		(current page.north west)
		{\includegraphics[height=16mm]{tukl_logo_left_padded.png}};
		\end{tikzpicture}
		\begin{center}
			\par\vspace{5cm}
			{\huge \textbf{3D Computer Vision
					\\[.2\baselineskip] 
					Exercise Sheet 1}}
					

			\par\vspace{2\baselineskip}
			{\large Marcel Brusius and Leif Eric Goebel}
			\par\vspace{3\baselineskip}
			{\large{ University of Kaiserslautern\\
					Departement of Mathematics\\		
				}}
				\par
				\vspace{3\baselineskip}
				Kaiserslautern, \today
				\thispagestyle{empty}
			\end{center}
		\end{titlepage}

	
	
	
	
%%%%%%%%%%%%%%%%%%Exercises%%%%%%%%%%%%%%%%%

%Exercise 1
\begin{Exercise}[Homography Definition]\ \\
	In the lecture a homography  was introduced as $h : \mathbb{P}^2 \rightarrow \mathbb{P}^2$. Define it for $h:\mathbb{P}^n \rightarrow \mathbb{P}^n$. How many degrees of freedom has the homography?
\end{Exercise}

\begin{Solution}[to Exercise 1]\ \\
	A Homography is an invertible, line-preserving projective mapping $h: \mathbb{P}^n \rightarrow \mathbb{P}^n$. It is represented by a square $(n+1) \times (n+1)$ matrix with $(n+1)^2-1$ degrees of freedom.
\end{Solution}

%Exercise 2
\begin{Exercise}\ \\
	A Homography between two images taken with the same camera can be used to compute the relative
	rotation $R_{rel}$ (in case the camera has only undergone a rotation between the shots). The relative rotation
	tells how the camera was placed between the two shots. For the following tasks assume a camera with
	given intrinsic parameters $(\alpha_x,\alpha_y, x_0, x_0, s)$. Use Matlab as a calculator and explain your intermediate
	steps.
\end{Exercise}

\begin{Solution}\ \\
	See file \verb|Homography.m|
\end{Solution}

%Exercise 3
\begin{Exercise}\ \\
	A homography between a plane in the world coordinate system and a camera image can be used to
	compute rotation R and translation t of the camera. Assume a camera with given $(\alpha_x, \alpha_y, x_0, y_0, s)$ and
	a homography H3, that was computed from the corners of a (fully visible) chessboard. The chessboard
	lies in the xy-plane of the world coordinate system centered around the origin. Compute R and t. Use
	Matlab as a calculator and explain your intermediate steps.
\end{Exercise}

\begin{Solution}
	See file \verb|Homography.m|. It is displayed in the console.
\end{Solution}

%Exercise 4
\begin{Exercise}\ \\
	\begin{itemize}
		\item [1.] Illustrate the meaning of $t = -RC$ in a camera pose $[R|t]$, i.e. from where to where does this vector point? Hint: What is linked by $[R|t]$? Try applying $[R|t]$ to the origin of the world. \\
		\item [2.] The third element of $t$ in exercise 3 might be negative. What does this mean in this particular case (consider the location of the chessboard corners)? Why can this happen?
	\end{itemize}
\end{Exercise}

\begin{Solution}
	\begin{itemize}
		\item [1.]	The vector $t = -RC$ illustrates the rotation and translation of the camera in world coordinates. The coordinates C are the of the camera in world coordinates. Applying R to them leads to the rotation. The $-$ in front of the vector makes it pointing from the camera c to the origin of the world. \\
		If we apply $[R|t]$ to the origin of the world we get again the origin of the world. \\
		\item[2.] The third element of the vector $t$ in the third exercise can be interpreted as the position of the camera relative to the XY-plane. If the third entry of $t$ is positive the camera is above the chessboard, whereat it is beneath the chessboard if the entry is negative. The corners of the chessboard remains the same.
	\end{itemize}

\end{Solution}

%Exercise 5
\begin{Exercise}\ \\
	Let $x_1, x_2, x_3 \in \mathbb{P}^2$ be three points on a line. Show that a homography H preserves this property. Hint: Use the implicit definition of a line $ax + by + c = 0$, thus $l = (a,b,c)$.
\end{Exercise}
By definition we have for a line 
\begin{equation*}
ax + by +c = 0.
\end{equation*}
In matrix vector notation we have for 
$l = 
\begin{pmatrix}
a \\
b \\
c
\end{pmatrix}$

\begin{equation*}
l^T x = 0 \text{ for } 
x = \begin{pmatrix}
x \\
y \\
1
\end{pmatrix}
\end{equation*}
Furthermore we have by definition, that each Homography $H$ is invertible and line preserving. We define 
\begin{equation*}
\tilde{x} := Hx.
\end{equation*}

Using the line-preserving property and the non-singularity of H we get
\begin{equation*}
0 = \underbrace{l^T H^{-1}}_{=:K} \tilde{x},
\end{equation*}
which is again a line equation.
\begin{Solution}
	
\end{Solution}


\newpage
\bibliographystyle{plain}
\bibliography{Literature}

\end{document}
